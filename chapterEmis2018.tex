\chapter[Emissions 2017]{OLD-OLD-OLD Emissions for 2017}
\label{ch:emis2017}

{\bf{Silke Gaisbauer, Robert Wankm\"uller, Bradley Matthews, Katarina Mareckova, Sabine Schindlbacher, Carlos Sosa, Melanie Tista and  Bernhard Ullrich}}
\vspace{30pt}

In addition to meteorological variability, changes in the emissions
affect the inter-annual variability and trends of air pollution,
deposition and transboundary transport.  
The main changes in emissions in 2017 with respect to previous years
are documented in the following sections.


%\section{Emissions for 2017}
%\label{Emis_2017}

The EMEP Reporting guidelines \citep{UNECE2014} requests all Parties
to the LRTAP Convention to report annually emissions and activity data of air pollutants
(\sox\footnote{``Sulphur oxides (\sox)'' means all sulphur compounds,
  expressed as sulphur dioxide (\soii), including sulphur trioxide
  (\soiii), sulphuric acid (\sulacid), and reduced sulphur compounds,
  such as hydrogen sulphide (H${_2}$S), mercaptans and dimethyl
  sulphides, etc.}, \noii\footnote{``Nitrogen oxides (\nox)'' means
  nitric oxide and nitrogen dioxide, expressed as nitrogen dioxide
  (\noii).}, NMVOCs\footnote{``Non-methane volatile organic
  compounds'' (NMVOCs) means all organic compounds of an anthropogenic
  nature, other than methane, that are capable of producing
  photochemical oxidants by reaction with nitrogen oxides in the
  presence of sunlight.}, \nhiii, CO, HMs, POPs,
PM\footnote{``Particulate matter'' (PM) is an air pollutant
  consisting of a mixture of particles suspended in the air. These
  particles differ in their physical properties (such as size and
  shape) and chemical composition. Particulate matter refers to:\\  
(i) ``PM$_{2.5}$'', or particles with an aerodynamic diameter equal to or
  less than 2.5 micrometers ($\mu$m);\\ 
(ii) ``PM$_{10}$'', or particles with an aerodynamic diameter equal to or
  less than 10 ($\mu$m).} and voluntary BC). Further, every four years,
projection data, gridded data and information on large point sources (LPS)
have to be reported to the EMEP Centre on Emission Inventories and Projections (CEIP).

%\subsection{Reporting of emission inventories in 2019}
\section{Reporting of emission inventories in 2019}

Completeness and consistency of submitted data have improved significantly since EMEP started collecting information on emissions. Data of at least 45 Parties each year were submitted to CEIP for the last seven years (see Figure~\ref{fig:CEIP1}). 45 Parties (88\%) submitted inventories\footnote{The original submissions from the Parties can be accessed via the CEIP homepage on \url{http://www.ceip.at/status_reporting/2019_submissions}.} in 2019; six Parties\footnote{Greece, Bosnia and Herzegovina, Kazakhstan, Liechtenstein, the Republic of Moldova, and Montenegro} did not submit any data and 39 Parties reported black carbon (BC) emissions (see section~\ref{sec:bc}). Although 2019 was no reporting year for large point sources (LPS), gridded emissions and projections, three Parties reported information on LPS, five Parties reported gridded data in new resolution, and twenty-five Parties reported projection data \citep{CEIP2019}.

The quality of the submitted data across countries differs quite significantly. By compiling the inventories, countries have to use the newest available version of the EMEP/EEA air pollutant emission inventory guidebook, which is the version of 2016 \citep{EmisInvGuide2016}. However, many countries still use the 2013 Guidebook \citep{EmisInvGuide2013} or older versions. Uncertainty of the reported data (national totals, sectoral data) is considered relatively high, the completeness of reported data has not turned out satisfactory for all pollutants and sectors either.

Detailed information on recalculations, completeness and key
categories, plus additional review findings can be found in the annual
EEA \& CEIP technical inventory review reports  \citep{CEIP2019} and
its
Annexes{\footnote{\url{http://www.ceip.at/review_proces_intro/review_reports}}. 

%\subsection{Black Carbon (BC) emissions}
\section{Black Carbon (BC) emissions}  
\label{sec:bc}

Over the last decade, black carbon (BC) has emerged as one of the most important anthropogenic air pollutants.  %% According to the latest independent inventory estimates with the GAINs model, global total anthropogenic emissions of BC were 7.2 Tg BC in 2010, with 4.16 Tg BC and 1.35 Tg BC originating from residential combustion and road transport sectors, respectively (\cite{Klimont:2017}). In their seminal review \cite{BonDohFah13} describe BC as ``a distinct type of carbonaceous material, formed only in flames during combustion of carbon-based fuels''. Black carbon is distinguished from other forms of carbon in atmospheric particulate matter (PM) e.g. organic carbon (OC) by its strong absorption of visible light, aggregate morphology, insolubility in water/common organic solvents, and that it is refractory (vaporization temperature ca. 4000K (\cite{BonDohFah13}).
Due to its distinct physical properties and its potential toxicity (\cite{Janssen:2012}) BC is a significant air pollutant in terms of both climate change and air quality. Given its absorption spectrum in the visible range, BC warms the atmosphere directly by absorbing solar radiation and, indirectly, by accelerating snow-/ice melt when deposited (\cite{BonDohFah13}). According to recent estimates, the direct radiative forcing effect of black carbon emissions during the first part of the industrial era may have been of the same magnitude as methane (\chiv) emissions (\cite{BonDohFah13,Wang2016}). Meanwhile, in terms of human health, epidemiological studies suggest that certain pulmonary and cardiovascular conditions are more strongly associated with exposure to BC rather than aggregate PM (e.g. \cite{Baumgartner:2014}). 
The emerging significance of BC is mirrored in developments in the international policy arena.
Since the Executive Body Decision 2013/04, Parties to the LRTAP Convention have been formally encouraged to
submit inventory estimates of their national BC emissions, and in 2015 the reporting templates were updated to include
BC data emissions. In addition to reporting under CLRTAP, EU member states are also encouraged to submit BC emissions
estimates as part of their emissions reporting under the National Emissions Ceilings (NEC) Directive (2016/2284/EU). Furthermore, in the context of the particularly acute impacts of BC in accelerating climate change in the Arctic (\cite{SanBerSal16}), ministers of the Arctic Council adopted the {\it Enhanced Black Carbon and Methane Emissions Reductions: An Arctic Council Framework Action} which committed the Arctic States (Canada, Denmark, Finland, Iceland, Norway, Russia, Sweden and United States of America) to develop and submit emissions inventories for BC and \chiv to the Council.

Under the auspices of the {\it EU Action on Black Carbon in the Arctic} (EUA-BCA) a technical report, {\it Review of Reporting
Systems for National Black Carbon Emissions Inventories}, was recently compiled. The report, which was co-led by CEIP
and will soon be published on the Action's website\footnote{\url{https://eua-bca.amap.no/}}, reviewed, {\it inter alia}, the level
of BC reporting under the LRTAP Convention. Despite a large number of Parties voluntarily reporting BC emissions, the review
revealed a number of shortcomings. As of 2018, nine Parties had not yet submitted BC emissions inventories to the
Convention. Furthermore, analysis of the emissions estimates which were reported revealed significant issues in terms
of consistency, completeness and comparability. As an example, Figure~\ref{fig:CEIP2} illustrates the varying level of reporting of
BC emissions for the sector {\it Residential combustion} N14 1A4bi. The review thus highlights that caution should be taken
when utilizing and/or analyzing BC emissions reported by the CLRTAP Parties.




Beyond this report CEIP continues to monitor and review the level of BC reporting by the Convention's Parties. Below a
brief overview of BC emissions estimates submitted by EMEP countries in 2019 is given.


Twenty-one countries (out of 39) submitted a complete time series of  national total BC emissions (1990-2017), 31 submitted a complete time series from 2000 onwards. Figure~\ref{fig:CEIP3} shows the 2000-2017 BC emissions trends for those 31 Parties. Most Parties (24) report a negative trend, with
the emissions of 22 Parties decreasing by 20\% or more. Six Parties report an increase in emissions when comparing the
2017 and 2000 estimates. 

Figure~\ref{fig:CEIP4}  gives an overview on the BC emissions reported for the year 2017. Thirty-eight of 39 reporting Parties reported
emissions for 2017. The United States reported BC emissions data in 2019, however, the most recent estimate is for the
year 2014. As the upper part of the graph illustrates, for the majority of these Parties, 2017 BC emissions constitute
between 10 and 20\% of the respective total \pmfine emissions. Indeed the median BC fraction based on reported BC and
\pmfine emissions lies at 15.01\%.


For more detailed information on BC consult the annual EEA \& CEIP technical inventory review report \citep{CEIP2019}.

%\subsection{Inclusion of the condensable component in PM emissions}
\section{Inclusion of the condensable component in PM emissions}
\label{sec:EmisSVOC}

The condensable component of particulate matter is probably the biggest single source of uncertainty in PM emissions.
%% The condensable component of particulate matter is released as a gas but forms particles when it is diluted and cools
%% down.
Currently the condensable component is not included or excluded consistently in PM emissions reported by Parties
of the LRTAP Convention. Also in the EMEP/EEA Guidebook \citep{EmisInvGuide2016} the condensable fraction is not consistently included or
excluded in the emission factors, but improvements are planned for the 2019 update
of the EMEP/EEA Guidebook. However, PM emissions reported by Parties to the LRTAP Convention are not directly
comparable at the moment with implications on the modeling of overall exposure to PM compliance with PM$_{2.5}$ emissions
reduction commitments.




Parties were asked to include a table with information on the inclusion of the condensable component in PM$_{10}$ and PM$_{2.5}$
emission factors for the reporting under the CLRTAP convention in 2019. This table has been added to the revised
recommended structure for IIRs\footnote{\url{https://www.ceip.at/ms/ceip_home1/ceip_home/reporting_instructions/annexes_to_guidelines/} }. Seventeen Parties
provided information on the inclusion of the condensable component in  PM$_{10}$ and PM$_{2.5}$ emission factors (Austria,
Belgium, Croatia, Estonia, Germany, Finland, France, Latvia, the Netherlands, Portugal, Romania, Slovakia, Slovenia,
Spain, Sweden, Switzerland  and United Kingdom)\footnote{Status 30 April 2019}. Eleven of these Parties provided the
information in the recommended format. Some Parties chose to report the information on an aggregated level. This
reporting is a first step towards a better understanding of the reported PM data. However, the reporting in 2019 showed
that in many cases Parties do not have the information if the PM emissions of a specific source category include the
condensable component. For the majority of the source categories of PM emission Parties either indicated that it is
``unknown'' if the condensable component is included in the PM emissions, or they provided no information or the
provided information was not clear. The status of inclusion or exclusion is best known for the emissions from road
transport. For example for ``1A3bi Road transport passenger cars'' ten of the twelve Parties that provided information
for this source category report emissions to be included and only two Parties state that the status of inclusion is
unknown.

Small-scale combustion sources make a notable contribution to total PM emissions. For all Parties that reported PM$_{2.5}$
emissions for ``1A4bi Residential: Stationary'' for the year 2017\footnote{ Status 2 May 2019} emissions from this
source category contributed 44\% to the national total PM$_{2.5}$ emissions. Small-scale combustion is one of the sources
where the inclusion of the condensable component has the largest impact on the emission factor. For example, for
conventional woodstoves, one of the most important categories in Europe, the emission factors excluding and including
the condensable fractions may differ by up to a factor of five \citep{DeniervanderGon2015}.  Here the
status of the inclusion was less clear. Of the thirteen Parties that provided information for ``1A4bi Residential:
Stationary'' three parties reported the condensable component to be included and three Parties to be excluded. The
other Parties reported ``unknown'', ``partially included'' or provided information on a more detailed level with
different status of inclusion (see Table~\ref{tab:CEIP1}).


As 2019 was the first year in which Parties were asked to provide information on the inclusion of the condensable
component, it is expected that the reporting will improve over the coming years, with more parties reporting the
information and with a higher quality of the reported information.




%\subsection[Comparison of 2016 and 2017]{Comparison of 2016 data (reported in 2018) and 2017 data (reported in 2019)}
\section[Comparison of 2016 and 2017]{Comparison of 2016 data (reported in 2018) and 2017 data (reported in 2019)}

The comparison of 2016 emissions (reported in 2018) and 2017 emissions
(reported in 2019) showed, that for 21 countries data changed by more
than 10\% for one or several pollutants (see Figure~\ref{fig:CEIP5} and Table~\ref{tab:emisdiffMAIN}-\ref{tab:emisdiffPM}).
These changes can be caused by real emission reductions or increases, or recalculations made by the respective country.

In five countries, both \nox and CO emissions changed by more than 10\%. For NMVOCs, emissions changed in two countries by more than 10\%. For \sox, emissions changed by more than 10\% in 14 countries, while for \nhiii the change of emission levels were less than 10\% in each country. Of the PMs, emissions changed by more than 10\% in ten countries for PM$_{2.5}$, in seven countries for PM$_{10}$ and in nine countries for PM$_{coarse}$\footnote{PM$_{coarse}$ emissions are not reported by Parties but calculated as difference between PM$_{10}$ and PM$_{2.5}$ emissions.} (see Figure~\ref{fig:CEIP5} and Table~\ref{tab:emisdiffMAIN}-\ref{tab:emisdiffPM}). The largest changes occurred in Belarus, Kyrgyzstan, Malta and Ukraine.




\clearpage
\bibliographystyle{copernicus}         % change bibliography-name after each
\renewcommand\bibname{References}      % bibliographystyle command!
\addcontentsline{toc}{section}{References}
\bibliography{Refs,EMEP_Reports}
