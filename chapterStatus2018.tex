\chapter[Status in 2017]{OLD-OLD-OLD Status of transboundary air pollution in 2017}
\label{ch:chapterStatus}

{\bf{Svetlana Tsyro, Wenche Aas, Sverre Solberg, Anna Benedictow, Hilde Fagerli and Thomas Scheuschner}}
\vspace{30pt}

This chapter describes the status of transboundary air pollution in 2017. A short summary of the meteorological conditions for 2017 is presented and the EMEP network of measurements in 2017 is briefly described. Thereafter, the status of air pollution and exceedances in 2017 is discussed.  

\section{Meteorological conditions in 2017}
\label{sec:meteo}
Air pollution is significantly influenced by both emissions and weather conditions. Temperature and precipitation are important factors and therefore a short summary describing the situation in 2017 as reported by the meteorological institutes in European and EECCA countries is given first.

The meteorological data to drive the EMEP MSC-W air quality model have been generated by the Integrated Forecast System model (IFS) of the European Centre for Medium-Range Weather Forecasts (ECMWF), hereafter referred to as the ECMWF-IFS model. In the meteorological community the ECMWF-IFS model is considered as state-of-the-art, and MSC-W has been using this model in hindcast mode to generate meteorological reanalyses for the year to be studied (Cycle 40r1 is the model version used for the year 2017 model run). Next section show temperature and precipitation in 2017 compared to the 2000-2016 average based on the same ECMWF-IFS model hindcast setup.

\subsection{Temperature and precipitation}
The mean temperature in 2017 was reported by the World Meteorological Organisation \citep{WMO1212:2018} as one of the three highest on record globally, the fifth highest in Europe. According to the Arctic Report Card 2017 \citep{Overland:ARC2017} October 2016 to September 2017 was the second warmest on record since 1900 in the Arctic.
The EMEP MSC-W model version rv4.33 has been used for the 2017 model
runs. The horizontal resolution is \resZO, with 20 vertical layers
(the lowest with a height of approximately 50 meters).
%as discussed in chapter~\ref{ch:ModelUpdates}.

 Meteorology, emissions, boundary conditions and forest fires for 2017
 have been used as input (for description of these input data see
 \citealt{Simpson_et_al:EMEP}). The meteorological input has been
 derived from ECMWF-IFS(cy40r1) simulations (\ref{sec:meteo}). The
 land-based emissions have been derived from the 2019 official data
 submissions to UNECE CLRTAP \citep{CEIP2019}, as documented in
 Chapter \ref{ch:emis2017}. Emissions from international shipping
 within the EMEP domain are derived from the CAMS global shipping
 emissions \citep{CAMSemis2019}, developed by the Finish
 Meteorological Institute (FMI). The forest fires emissions are taken
 The Fire INventory from NCAR (FINN) \citep{FINNIGAN1990}, version 5.
 For more details on the emissions for 2017 model run see Chapter
 \ref{ch:emis2017} and Appendix \ref{ch:appx_emis_2017}.

 Preliminary simulations for 2018 have been performed with the same
 EMEP MSC-W model version ~(rv4.33), driven with 2018 meteorological
 input (also derived from ECMWF-IFS cy40r1), and used the same emissions
 (anthropogenic and forest fires) as in the 2017 run. Climatological
 means were used for boundary conditions. No evaluation of the 2018
 results have been made as EMEP observational data for 2018 were not
 available. The model results for 2018 can be downloaded from the EMEP
 webpage (\url{http://www.emep.int}).

 Trend runs with the EMEP MSC-W model have been performed for the
 period of 2000--2016, using meteorological data and emissions for the
 respective years. The land-based emissions for 2000--2016 were
 derived from the 2019 official data submissions to UNECE CLRTAP
 \citep{CEIP2019}, and the international shipping emissions were
 derived from the CAMS global shipping emission dataset
 \citep{CAMSemis2019,ECCAD}, produced by FMI using AIS (Automatic
 Identification System) tracking data (see also Appendix
 \ref{ch:appx_emis_trends}). FINNv5 forest fire emissions have been
 used for corresponding years in the runs for 2002-2016, whereas for
 2000 and 2001 average emissions over the 2005-2015 period have been
 used. Note that the \sox emissions from the eruption of the Grimsvotn
 volcano in 2011 have deliberately been excluded, since the model
 cannot accurately simulate their dispersion as their intrusion
 occurred above the model's top layer.



\section{Air pollution in 2017} %Hilde and Sverre to rewrite

\subsection{Ozone}
\label{O3MAX}

The ozone observed at a surface station is the net result of various physio-chemical processes; surface dry deposition and uptake in vegetation, titration by nearby \nox emissions, regional photochemical ozone formation and atmospheric transport of baseline ozone levels, each of which may have seasonal and diurnal systematic variations. Episodes with elevated levels of ozone are observed during the summer half year when certain meteorological situations (dry, sunny, cyclonic stable weather) promotes the formation of ozone over the European continent. 

Figure~\ref{indicators} shows various modelled ozone metrics for 2017 with the corresponding measured metrics based on the EMEP measurement sites plotted on top of the maps. Figure~\ref{indicators_airbase} show similar plots with data from Airbase measurement sites. Note that most of the EMEP sites are also classified as Airbase sites and thus included in Figure~\ref{indicators_airbase} as well. Only stations located below 500 metres above sea level were used in this comparison to avoid uncertainties related to the extraction of model data in regions with complex topography. The maps show a) the mean of the daily max concentration for the 6-months period April-September, b) SOMO35 (= Sum of Ozone Means Over 35 ppb), c) AOT40 for forests (= Accumulated Ozone exposure over a Threshold of 40 ppb) for the 6-months period April-September using the hours between 08 and 20 and d) POD$_1$ for forests (= Phytotoxic Ozone Dose above a threshold 1 mmol m$^{-2}$) (only for Figure~\ref{indicators}). POD$_1$ could not be calculated from the ozone monitoring data directly and are thus not given in plot d).

The mean daily max O$_3$, SOMO35 and AOT40 all show a distinct gradient with levels increasing from north to south, a well established feature for ozone in general reflecting the dependency of ozone on the photochemical conditions. Ozone formation is promoted by solar radiation and high temperatures. The highest levels of these ozone metrics are predicted over the Mediterranean Ocean and in the southeast corner of the model grid. The measurement network are limited to the continental western part of the model domain with no valid data in Belarus, Ukraine, Turkey or the area further east.

For the region covered by the monitoring sites, the pattern with increased levels to the south with maximum levels near the Mediterranean is seen in the measurement data as well as the model. The geographical pattern in the measured values are fairly well reflected by the model results for all these three metrics. In particular, the modelled mean daily max for the summer half year agrees very well with the measured values. Particularly high levels are predicted by the model in the southeast, but due to the lack of monitoring sites these levels could not be validated. 

\clearpage


A good agreement between modelled and observed levels of SOMO35 and AOT40 is also seen from Figure~\ref{indicators} and Figure~\ref{indicators_airbase}. It should be noted that the O$_3$ metrics such as AOT40 are very sensitive to the calculation of vertical O$_3$ gradients between the middle of the surface layer and the 3m height used for comparison with measurements \citep{Tuovinen:EP2007} and thus more difficult to compare with measurement data than e.g. the mean daily maximum. Indeed, the formulation we use \citep{Simpson_et_al:EMEP} is probably better suited to a first model layer of 90m height (since we equate the centre of this, ca. 45m, with a `blending-height') than to a first level of 50m height (as used throughout this report). 

The modelled POD$_1$ pattern differs from the other metrics reflecting the influence of additional parameters such as plant physiology, soil moisture etc., and is a metric more indicative of the direct impact of ozone on vegetation than e.g. AOT40. The POD$_1$ field could however not be validated by the EMEP ozone measurement data alone. 

SOMO35 is an indicator for health impact assessment recommended by WHO, and the results given in Figure~\ref{indicators} and Figure~\ref{indicators_airbase} indicates that the health risk associated with surface ozone increased towards southern Europe in 2017. SOMO35 is a health risk indicator without any specific threshold or limit value.

AOT40 and POD$_1$ are indicators for effects on vegetation. UN-ECE's critical level for forests is 5000 ppb hours, and the measurements given in Figure~\ref{indicators} and Figure~\ref{indicators_airbase} indicate that this level was exceeded in most of the European continent in 2017 whereas it was not exceeded in most of Scandinavia and the British Isles. As mentioned, the model predicts larger areas with exceedances than the measurements. For POD$_1$ the limit value depends on the species and Mills et al (2011) give a value of 4 mmol m$^{-2}$ for birch and beech and 8 mmol m$^{-2}$ for Norway spruce. The results in Figure~\ref{indicators} indicate that both these limit values were exceeded in most of Europe. The modelled levels of POD$_1$ could, however, not be validated by observations. 

A more detailed comparison between model and measurements for ozone for the year 2017 can be found in \cite{WEB2019:Eval}.



\subsubsection{Ozone episodes in 2017}
The surface ozone levels are closely connected to the weather conditions and in particular to the temperature. As shown in Figure~\ref{fig:temp2017-avMET} a) the 2m summer temperatures were lower than normal in Germany, Poland and Scandinavia in 2017 and slightly above the normal in the UK and France. As discussed above, a stronger positive temperature anomaly was seen in the Mediterranean, in particular in Spain and Portugal. 

This is reflected in the occurrence of ozone episodes. Relatively few marked episodes was experienced in central and northern Europe whereas in the south, the Po Valley and the Iberian peninsula experienced a number of episodes of smaller regional extent. In the following, three episodes affecting larger parts of the European continent are presented briefly - one in the end of May, one in the last part of June and the last one in the beginning of August. 


\subsubsection{28 - 29 May}
This episode was linked to a high pressure located over central Europe and low pressures west of Spain/France setting up a southerly flow of warm air into central Europe. The episode lasted only a couple of days and affected mainly the Netherlands, Germany and northern Italy. The episode peaked on 29 May when several sites crossed EU's information threshold of 180 \ug and two sites experienced ozone levels above 200 \ug. On 30 May the warm air mass had moved to the east with lower ozone levels. As shown in Figure~\ref{fig:o3_20170528} and Figure~\ref{fig:o3_20170529} the model captures the area of elevated ozone very well but apparently underpredicts the daily maximum ozone in some areas, particularly Belgium and the Netherlands. 
\clearpage

%\clearpage


\clearpage
\bibliographystyle{copernicus}         % change bibliography-name after each
\renewcommand\bibname{References}      % bibliographystyle command!
\addcontentsline{toc}{section}{References}
\bibliography{Refs,EMEP_Reports}
