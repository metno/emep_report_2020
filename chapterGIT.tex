\chapter{GIT usage}

This report is usable both from the web-interface overleaf \url{https://www.overleaf.com/project/5ebea6ecfa55c300015e6073} and directly from github: \url{https://github.com/metno/emep_report_2020}

\section{Report usage from overleaf}

When starting, get the latest version from github
\begin{itemize}
    \item Press Menu -> (Sync) GitHub
    \item Select "Pull changes from github" (get all changes from other contributors, not always needed)
    \item Start making your changes
\end{itemize}

When you want to make the results available for others to see/use:
\begin{itemize}
    \item Press Menu -> (Sync) GitHub
    \item Select "Pull changes from github" (get all changes from other contributors, not always needed)
    \item Select "Push Overleaf changes to github" (github users will see all changes visible on overleaf)
\end{itemize}


\section{Report usage from GitHub}

Download the report once:
\begin{verbatim}
    git clone git@github.com:metno/emep_report_2020.git
\end{verbatim}
All further work will be done in the directory emep\_report\_2020.

To get the latest version of the report, run from within above directory:
\begin{verbatim}
    git pull
\end{verbatim}

To create a local version of your changes add and commit the to your local repository:

\begin{verbatim}
    git add file1.tex file2.tex
    git commit -m 'changed a bit'
\end{verbatim}

Check regularly if you haven't forgotten to add any files to your local repository:
\begin{verbatim}
    git status
\end{verbatim}

Make your local repository available to everybody else. (Ensure you are working on the latest version. And if conflicts, you should fix them and prepare a new version):
\begin{verbatim}
    git pull # accept merging changes
    git push
\end{verbatim}

