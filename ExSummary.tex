\chapter*{OLD-OLD-OLD Executive Summary}
%HF REARRANGE, shorten, start with the interesting stuff....Also rewrite the intro-part
This report presents the EMEP activities in 2018 and 2019 in relation to transboundary
fluxes of particulate matter, photo-oxidants, acidifying and
eutrophying components, with focus on results
for 2017. It presents major results of the activities related to
emission inventories, observations and modelling. The report also
introduces specific relevant research activities addressing EMEP key
challenges, as well as technical developments of the observation and
modelling capacities.\\



\noindent
\textbf{Measurements and model results for 2017}\\ %Wenche has updated except on ozone
In the first chapter, the status of air pollution in 2017 is presented, combining 
meteorological information and emissions with numerical simulations using the EMEP MSC-W model 
together with observed air concentration and deposition data.

Altogether 35 Parties reported measurement data for 2017, from 171 sites in total. 
Of these, 139 sites reported measurements of inorganic ions in precipitation and/or 
main components in air; 75 of these sites had co-located measurements in both air and 
precipitation. The ozone network consisted of 139 sites, particulate matter was measured at 
69 sites, of which 50 performed measurements of both \PM[10] and \PM[2.5]. 
In addition, 45 sites reported at least one of the components required in the advanced 
EMEP measurement program (level 2). A complete aerosol program was implemented at 8 sites, 
while only a few sites provided the required oxidant precursor measurements.

The mean daily max O$_3$, SOMO35 and AOT40 all show a distinct gradient with levels increasing from north to south, a well established feature for ozone reflecting the dependency of ozone on the photochemical conditions. The geographical pattern in the measured values is fairly well reflected by the model results for all these three metrics. In particular, the modelled mean daily max for the summer half year agrees very well with the measured values except for an underestimation in a few regions, mainly in the Mediterranean. Particularly high levels are predicted by the model in the south-east, but due to the lack of monitoring sites these levels could not be validated.

The model results and the observations agree quite well on the geographical distribution of annual mean \PM[10] and
\PM[2.5], with concentrations below 2-5 \ug in northern Europe,
increasing to 5-15 \ug in central Europe and further south. The
regional background PM is fairly homogeneous over most of central and
western Europe, with somewhat elevated \PM[10] and \PM[2.5] levels of
15-20 \ug modelled for the Po Valley, the Benelux countries, and also observed
in Poland, Czechia, Hungary and Spain. On average, the model
underestimates the observed 2017 annual mean \PM[10] and \PM[2.5] by
22\% and 19\% with annual mean spatial correlation coefficients of 0.76 and 0.81,
respectively.

Due to meteorological conditions, the annual mean \PM[10] and \PM[2.5]
concentrations were 5 to 20\% lower in 2017 compared to the 5-year
(2012-2016) mean over central, eastern and south-eastern Europe, and
the North Atlantic coast, with the largest negative anomalies of
20-30\% seen over northern and north-western Europe. Due to the combined
effect of meteorology and emission changes, annual mean 
\PM[10] and \PM[2.5] in 2017 were considerably lower compared to the average
levels in the 2000s, by 5-20\% over Spain, Portugal and Italy and most
of Russia, and by as much as 20-35\% in many parts of northern,
western and central Europe. In addition to emission reductions, 
2017 was a meterologically favorable year in terms of air pollution
removal by precipitation.
\\

\noindent
\textbf{Exceedances and pollution episodes in 2017}\\  
In 2017, relatively few high ozone episodes were
experienced in central and northern Europe whereas southern Europe, in particular the Po Valley and the Iberian
Peninsula, experienced a number of episodes of smaller regional extent. An intense heat wave (named Lucifer) struck parts of southern Europe (south-east France, Italy, the Balkans) in early August, described as the worst heat wave since 2003 here. The highest ozone level observed, 119.5 ppb (239 \ug), was just below EUs alert level of 240 \ug, and was recorded at the rural background site Parco La Mandria in north-west Italy (data from the EEA data base).  The EMEP MSC-W model reproduce the observed geographical extent of the episode very well, but underpredicts the peak values in many areas.

Model results and EMEP observational data show that in 2017, the
annual mean \PM[10] and \PM[2.5] concentrations were below the EU
limit values for all of Europe. However, exceedences of the  Air Quality Guideline (AQG)
recommended by WHO (10 \ug) in the annual mean of \PM[2.5] were
observed at ten sites.

Exceedance days for \PM[10] were observed at 35 out of 58
sites, but no violations of the \PM[10] EU limit value (more than 35
exceedance days) were registered. 18 sites had more than 3
exceedance days, the recommended AQG by WHO.
\PM[2.5] concentrations exceeded the WHO AQG value at 35 out of 46 stations in 2017 
(on more than 3 days at 26 sites).

The largest PM pollution episodes occurred in January-February 
and affected air quality in many European countries. The
timeseries of modelled and observed chemical composition of \PM[2.5]
at selected sites in France, Poland and Czechia (and also
modelled \PM[10] chemical composition for several sites in central and
eastern Europe), during the January-February 2017 episodes indicate
a diversity of emission sources causing the episodes at
different locations.

Critical loads (CL) for eutrophication were exceeded in virtually all countries in 2017, in about 63.9\% of the ecosystem area, and the European average exceedance was about 277 eq ha$^{-1}$yr$^{-1}$. The highest exceedances are found in the Po Valley in Italy, the Dutch-German-Danish border areas and in north-east Spain. In contrast, critical loads of acidity were not exceeded in most of Europe. Hot spots of exceedances can be found in the Netherlands and its border areas to Germany and Belgium, and some smaller maximum in southern Germany and the Czechia. In Europe as a whole, acidity exceedances in 2017 occur in about 5.5\% of the ecosystem area, and the European average exceedance is about 32.4 eq ha$^{-1}$yr$^{-1}$.\\

\newpage

\noindent
\textbf{Status of emissions}\\%Agnes 
In 2019, 45 out of 51 Parties (88$\%$) submitted emission inventories to the EMEP Centre on Emission Inventories and Projections (CEIP). The quality of reported data differs significantly across countries, and the uncertainty of the data is considered to be relatively high.

%Over the last decade, black carbon (BC) has emerged as one of the most important anthropogenic air pollutants.
%Since the Executive Body Decision 2013/04, Parties to the LRTAP Convention have been formally encouraged to
%submit inventory estimates of their national BC emissions, and in 2015 the reporting templates were updated to include
%BC data emissions.
Under the auspices of the {\it EU Action on Black Carbon in the Arctic} a technical report  %{\it Review of Reporting Systems for National Black Carbon Emissions Inventories},
was recently compiled. The report reviewed, inter alia, the level of BC reporting under the LRTAP Convention.
Despite a large number of Parties voluntarily reporting BC emissions, the review
revealed a number of shortcomings. As of 2018, nine Parties had not yet submitted BC emissions inventories to the
Convention. Furthermore, significant issues in terms
of consistency, completeness and comparability were found in the reported emissions.
%In 2019, 21 countries (out of 39) submitted a complete time series of  national total BC emissions (1990-2017), whilst 31 submitted a complete time series from 2000 onwards. Most Parties (24) report a negative trend, with
%the emissions of 22 Parties decreasing by 20\% or more. Six Parties report an increase in emissions when comparing the
%2017 and 2000 estimates.
For the majority of the Parties which reported
emissions for 2017, BC emissions constitute
between 10 and 20\% of the respective total \pmfine emissions.
%The median BC fraction based on reported BC and
%\pmfine emissions lies at 15.01\%.


The condensable component of particulate matter is probably the largest single source of uncertainty in PM emissions. Currently the condensable component is not included or excluded consistently in PM emissions reported by Parties
of the LRTAP Convention. Parties were asked to include a table with information on the inclusion of the condensable component in PM$_{10}$ and PM$_{2.5}$
emission factors for the reporting under the CLRTAP convention in 2019. This table was added to the revised
recommended structure for informative inventory reports. This year, 17 Parties
provided information on the inclusion of the condensable component.
%This
%reporting is a first step towards a better understanding of the reported PM data.
However, the reporting in 2019 showed
that in many cases Parties do not have information on whether or not the PM emissions of a specific source category include the
condensable component. The status of inclusion or exclusion is best known for the emissions from road transport, whilst it is less clear for small-scale combustion sources.
%As 2019 was the first year in which Parties were asked to provide information on the inclusion of the condensable
%component, it is expected that the reporting will improve over the coming years, with more parties reporting the
%information and with a higher quality of the reported information.

2017 was the first year with reporting obligation of gridded emissions in  0.1{\degrees}$\times$0.1{\degrees} lon\-gi\-tude/la\-ti\-tude resolution. 
Until June 2019, thirty of the 48 countries which are considered to be part of the EMEP area reported sectoral gridded emissions in this resolution. %One country reported only gridded national total values (instead of sectoral data).
%The majority of gridded sectoral emissions in 0.1{\degrees}$\times$0.1{\degrees} lon\-gi\-tude/la\-ti\-tude resolution have been reported for the year 2015 (29 countries). For the year 2017, gridded sectoral emissions have been reported by four countries.%Reported gridded sectoral data cover less than 20\% of the grid cells within the geographical EMEP domain.
For remaining areas missing emissions are gap-filled and spatially distributed using expert estimates. This year CEIP also performed gap-filling and gridding for the whole time series from 1990 to 2017 in 0.1{\degrees}$\times$0.1{\degrees} lon\-gi\-tude/la\-ti\-tude resolution on GNFR sector level for the main pollutants, and from 2000 to 2017 for PMs.
%In addition, gap-filling and gridding for BC was done for the first time, but only for the year 2017.
Emissions from international shipping in different European seas were updated based on the CAMS global shipping emission dataset for the years 2000 to 2017, provided via ECCAD {\it CAMS\_GLOB\_SHIP}. Shipping emissions from 1990 to 1999 were estimated using CAMS global shipping emissions for 2000, adjusted with trends for global shipping from EDGAR v.4.3.2.

%The development in emissions in the eastern and western parts of the EMEP area seems to follow different patterns. While emissions of all pollutants in the western part of the EMEP domain are slowly decreasing, emissions of all pollutants in the eastern part of the EMEP domain have increased since the year 2000. The emissions in western parts of the EMEP area are mostly based on reported data, while the emissions in eastern parts often are based on expert estimates (with larger uncertainty).
%From 2000 to 2017, the total change in emissions for the EMEP domain has been: NO$_x$ (-6$\%$), NMVOCs (-5$\%$), SO$_2$ (-29$\%$), NH$_3$ (+31$\%$), PM$_{2.5}$ (+10$\%$), PM$_{10}$ (+17\%), PM$_{coarse}$ (+33$\%$) and CO (-9$\%$).

The 1999 Gothenburg Protocol lists emission reduction commitments of NO$_x$ ,
SO$_x$, NH$_3$  and NMVOCs for most of the Parties to the LRTAP Convention for the year 2010. These commitments should not be exceeded in 2010 nor in subsequent years.
%In 2012, the Executive Body of the LRTAP Convention decided that adjustments to emission inventories may be applied in some circumstances. From 2014 to 2018, adjustment applications of nine countries have been accepted and therefore these approved adjustments have to be subtracted for the respective countries when compared to the GP targets. In April 2019, the
%Netherlands submitted a new adjustment application, which will be approved most likely later this year. Further,  the reporting guidelines specify that some Parties within the EMEP region may choose to use the national emission total calculated on the basis of fuels used in the geographic area of the Party as a basis for compliance with their respective emission ceilings.
%However,
When considering only reported data, approved adjustments and fuel use data of the respective countries, it can be seen that the Netherlands and USA had not reduce their NMVOC emissions according to the Gothenburg Protocol requirements, and that Croatia, Germany, Norway and Spain are above their Gothenburg Protocol ceilings for \nhiii. In terms of \nox emissions, Norway exceeded its ceilings.\\



\noindent
\textbf{Condensable organics; issues and implications for EMEP calculations and source-receptor matrices}\\
Estimates of PM and NMVOC emissions as currently provided by Parties have a number
of major uncertainties, and there is a clear need for clarification and
standardisation of the methods used to define and report PM emissions,
also concerning the fraction of PM that is primary organic aerosol (POA).
For example, emissions from residential wood-burning in Europe represent
around 50\% of Europe's POA emissions, and they dominate wintertime POA
sources, but several studies show that the definitions behind national
emission estimates are inconsistent in their treatment of condensable
organic compounds.  A new bottom-up emissions inventory for OA was
implemented for this study, taking account of condensable organics.
For some countries (e.g. NO, DK) the bottom-up and EMEP estimates of
\pmfine emissions are comparable, but for others (e.g. FI, SE) the expert
estimates are far higher than the reported emissions.  The new inventory
gave improved model performance for organic aerosol and thus \pmfine,
especially in wintertime.  We show that source-receptor calculations
are also sensitive to these uncertainties, both for \pmfine and
especially for organic aerosol contributions.  Such inconsistencies
pose grave problems for the modeling of \pmfine  and for any analysis of
emission control strategies or cost-benefit analysis. In the worst case
these problems might lead to wrong priorities of measures.  A review
and harmonisation of methods for PM and POA emission inventories is
recommended.\\


\noindent
\textbf{The EMEP Intensive Measurement Period (EIMP) 2017/18: Equivalent Black Carbon (EBC) from fossil fuel and biomass burning sources}\\
In this report we present results from the ongoing analysis of data from the EMEP IMP 2017/18. We present source apportionment of equivalent black carbon into fossil and biomass fractions (\EBCff and \EBCbb, respectively), using the aethalometer model and positive matrix factorization (PMF). According to the aethalometer model, \EBCbb represents between close to zero (e.g. Beirut, Lebanon) and just over 50 \% (e.g. Beograd, Serbia) of background EBC. However, this model requires a priori knowledge of the aerosol {\AA}ngstr{\"o}m exponent (AAE), and results from the aethalometer model vary widely depending on the input AAEs. Using a new application of PMF to aethalometer data, we were able to identify \EBCff and \EBCbb  results without input AAEs (rather AAEs are an output derived from factor profiles).

EMEP MSC-W model calculations were performed for the time period of the EMEP EIMP, using several sources of EC emission data, including the reported EMEP EC emissions.
%e.g. 1) the reported EMEP EC emissions and 2) EC emissions from a data set developed by TNO (CAMS\_2015\_RWC).
The resulting modelled EC concentrations and the share of EC concentrations  from biomass burning (EC$_{bb}$) and fossil fuel (EC$_{ff}$) sources were compared to the preliminary data available from the EIMP (EBC and biomass burning fractions from PMF). The results suggest that the EC emissions are somewhat low (or the spatial distributions are erroneous) for this winter period, especially in the reported EMEP EC emission inventory.  All the model results show reasonable agreement with observations at rural sites, whilst there is no correlation between model results and observations at urban sites (and even anti-correlation when reported EMEP EC emissions are used). 

The fractions of EC emissions from biomass burning sources versus fossil fuel are very different in the reported EMEP emissions and in the emission data set developed by TNO (CAMS\_2015\_RWC), resulting in substantially different modelled EC$_{bb}$/EC$_{ff}$ concentration fractions. Model calculations based on reported EMEP EC emissions are in reasonable agreement with the PMF values for biomass burning fractions, whilst model results based on CAMS\_2015\_RWC give consistently higher biomass burning fractions than PMF. Given that the reported EC emissions might be somewhat low, the proportion of different EC sources in the EMEP emission data could be approximately correct for the wrong reasons, as the emissions from different sources (with completely independent emission factors) would have to increase proportionally to keep the biomass burning fraction about the same.

Only a subset of the EIMP data has been used in this analyses, and measurement data will become available for more sites in the near future. Further investigations, including in depth analyses of model results at the different rural and urban EIMP sites and spatial distribution of emissions for different emission sectors, are needed to determine the validity and possible implications of these preliminary results.
\\


\noindent
\textbf{The EMEP trend interface}\\ %HF to write
A new trend interface is under development at MSC-W (available at \url{http://aerocom.met.no/trends/EMEP/}). The trend interface is designed for visualization of the
long-term modelling results at all EMEP sites that have reported observations to EMEP/CCC. A range of new functionalities have been implemented in the interface since last year, the most important being inclusion of EMEP observations and model evaluation statistics. The interface has been extended to include more
species, and now visualizes data for ozone, \PM[10] and \PM[2.5]. Furthermore, the impacts from different
emission sectors on \PM[10] and \PM[2.5] concentrations are vizualised
and a number of other technical facilities have been introduced.\\

\noindent
\textbf{Evaluation of the gridded EMEP 0.1{\degrees}$\times$0.1{\degrees} emissions using modelling}\\%HF to write
EMEP MSC-W model results using the EMEP 0.1{\degrees}$\times$0.1{\degrees} resolution emissions have been compared to model results using the older 50km $\times$ 50km resolution and to model results using CAMS-REG-AP - a widely used set of fine resolution emissions (\ensuremath{0.1^{\circ} \times 0.05^{\circ}}) developed by TNO. The three sets of model results
have been compared to AirBase observations for each country individually, focusing on the spatial distribution of the results. 
The largest improvement in going from 50km $\times$ 50km resolution to  0.1{\degrees}$\times$0.1{\degrees} resolution is seen for NO$_2$, which can be explained by the high correlations between emissions and surface concentrations of NO$_2$. Interestingly,  for NO$_2$ the model results using the EMEP 0.1{\degrees}$\times$0.1{\degrees} resolution emissions have higher (or similar) spatial correlation compared to observations for most countries than model results based on CAMS-REG-AP, suggesting that the gridding performed by the countries are superior to the gridding done for CAMS-REG-AP. This may not be surprising, as the gridding done by the countries in most cases are based on national data, that are probably better than the European-wide proxies used for CAMS-REG-AP. For some countries, the model runs with fine resolution EMEP emissions showed substantially worse
correlation to observations than the CAMS-REG-AP fine resolution emissions. For these countries it would be worthwhile looking further
into the methodology used for spatial distribution of the emissions.\\


\noindent
\textbf{Baltic Sea shipping}\\ %Joffen and Michael
As part of the EU Interreg project EnviSum the effects of emissions from
Baltic Sea shipping on air pollution and health have been calculated, and
the results published in two journal papers. A resume of the
papers is given in this report. We find that the implementation of the
stricter SECA regulations from January 2015 has been successful in
reducing sulphur emissions from shipping.
As a result, \chem{PM_{2.5}} concentrations, in particular in coastal zones,
have been reduced. A large portion of the population in the Baltic Sea
region lives in the coastal zones. The stricter SECA regulations have 
alleviated the health burden in the region by reducing the mortality and
morbidity from Baltic Sea shipping by about one third. The main source of
\chem{PM_{2.5}} from the Emission Control Areas in the Baltic Sea (and the
North Sea) is now \chem{NO_x}, and the resulting health effects are still
significant. \chem{NO_x}  will be regulated from 2021 in the region, but
only for new ships, resulting in only a gradual decrease in emissions.
\\





\noindent
\textbf{Model improvements}\\ %HF added some text
%Most of the changes made in the EMEP MSC-W model since last year have been concerned with improvements to
%the model code and usability, and these have had little impact on model results.
%These improvements include several updates and bug-fixes to the chemical scheme, 
%improved compatibility between the older SNAP and new GNFR emission sectors, updated
%land-cover database and improved handling of WRF and AROME meteorology.
%One major change did occur, however, and that concerns the treatment of photosynthetically
%active radiation (PAR) in the model, which impacts both biogenic VOC emissions and ozone flux estimates.
%The changed radiation scheme seems to mainly impact POD$_1$ estimates for forests (now reduced), with
%only small changes in POD$_3$ for crops or ozone concentrations. 
The model version used for reporting this year has some
significant changes since the rv4.17a documented last year. A new gas-phase chemical mechanism has been introduced (EmChem19), which is a
substantial revision of the EmChem16 scheme used previously. The reaction rates in EmChem19 are updated to be
consistent with the latest recommendations from the IUPAC. In addition to these updates some new gas-phase reactions have
been added and a few new chemical species have been included in the chemical mechanism,
in order to be more consistent with the
Master Chemical Mechanism (MCM). An error in the calculations of photosynthetically active
radiation (PAR) has been fixed, impacting mainly the ozone uptake for forests.

%The latest version of the Equlibrium Simplified Aerosol Model V4 (EQSAM4clim), has been implemented in the EMEP MSC-W model as one
%of the alternative schemes to calculate gas/aerosol partitioning.
%First tests show that the results from EQSAM4clim are
%very similar to those obtained using more computionally expensive MARS thermodynamic scheme. Furthermore, the EQSAM4clim scheme allows
%to complete the thermodynamic equilibrium with sea salt and mineral dust, which is expected to further
%improve EQSAM4clim performance. 

The latest version of the Equlibrium Simplified Aerosol Model V4 (EQSAM4clim), has been implemented in the EMEP MSC-W model as one
of the alternative schemes to calculate gas/aerosol partitioning. First tests show that the results from EQSAM4clim are very similar, or slightly better, 
than those obtained with the MARS thermodynamic scheme. The advantage of EQSAM4clim is that the scheme allows completing the 
thermodynamic equilibrium with missing cations and anions from sea salt and mineral dust, which is anticipated to further
improve EQSAM4clim performance.

In addition, a number of technical improvements with respect to flexibility and usability of the model have been made. 
\\

%\noindent
%\textbf{EQSAM4}\\ %Combine with above and shorten
%The latest version of the Equlibrium Simplified Aerosol Model V4,
%namely EQSAM4clim, has been implemented in the EMEP MSC-W model as one
%of the alternative schemes to calculate gas/aerosol partitioning. The
%EQSAM4clim is designed to combine computational efficiency with
%accuracy and flexibility. The main results from EQSAM4clim evaluation
%against observations and comparison with the MARS model (which is
%presently in operational use in the EMEP MSC-W model) are
%presented. Those first tests show that the results from EQSAM4clim are
%very similar (to those obtained using MARS thermodynamic scheme. Given
%that, the EQSAM4clim scheme is considered to be a good candidate for
%future use in the EMEP MSC-W model, which will also allow us to
%complete the thermodynamic equilibrium the missing cations and anions
%from sea salt and mineral dust. Further testing of EQSAM4clim is still
%needed before it could be employed in EMEP status and source-receptor
%calculations.\\



\noindent
\textbf{Development in the monitoring network and database infrastructure}\\
The last chapter of the report presents the implementation of the EMEP monitoring strategy 
and general development in the monitoring programme including data submission. There are 
large differences between Parties in the level of implementation, as well as significant changes in the national activities during the period 2000-2017. With respect to the requirement 
for level 1 monitoring, 40\% of the Parties have had an improvement since 2010, 
while 33\% have reduced the level of monitoring. For level 2 monitoring there has been 
a general positive development in recent years. However, only few sites have a complete measurement program.
\\
The complexity of data reporting has increased in recent years, and it is therefore now mandatory for the
data providers to use the submission and validation tool when submitting data to EMEP
to improve the quality and timeliness in the data flow. 
There is a need for improvements in the reporting, as only half of the data 
providers use the submission tool, and less than 60\% report within the deadline of 31 July.
\\

