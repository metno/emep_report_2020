\chapter[Model updates]{OLD-OLD-OLD Updates to the EMEP MSC-W model, 2018-2019}
\label{ch:ModelUpdates}

%% authors? added all names seen in Log.changes

{\bf{David Simpson, Robert Bergstr\"om, Svetlana Tsyro and Peter Wind}}
\vspace{30pt}

This chapter summarises the changes made to the EMEP MSC-W  model
since \citet{R2018:ModDev}, and along with changes discussed in
\citet{R2013:ModDev,R2014:ModDev,R2015:ModDev,R2016:ModDev,R2017:ModDev}, updates
the standard description given in \citet{Simpson_et_al:EMEP}. The
model version used for reporting this year is denoted
rv4.33, which has some significant changes since the
rv4.17a documented last year.
Table~\ref{tab:Updates} summarises the changes made in the EMEP
model since the version documented in \citet{Simpson_et_al:EMEP}.

%Versions:
%
%  rv4.33 used  for R2019
%  rv4.33 - open source June 2019
%  rv4.32 used for EMEP course, April 2019
%  rv4.17a used for R2018 runs  (July-ish?)
%  rv4.17 released 26/2/2018
%  rv4.16 interim 21/12/2017 - used for N2O5 paper, wheat calculations
%  rv4.15 released 8/9/2017

\section{Overview of changes} 

\begin{itemize}

\item
EmChem19 -- a new gas-phase chemical mechanism has been
introduced (Bergstr{\"o}m et al., 2019), updating the EmChem16x scheme used previously. See
Sect.~\ref{sec:EmChem19}.

\item
CRI v2.2a-emep -- the Common Reactive Intermediates (CRI) mechanism \citep{Jenkin2008}
used
in research versions of the EMEP model has been updated and a new isoprene scheme for the
CRI scheme has been implemented in the model \citep{JenkinCRI2019,McFiggans:2019}.

\item
The carbon-bond scheme used in research versions of the EMEP model has been updated from
CB05 \citep{Yarwood_et_al:2005} to CB6r2
\citep[]{LueckenAE2019}.

\item
Two new schemes for handling secondary organic aerosol (SOA) have been
added to the model -- a new volatility basis set (VBS) scheme from
\citet{Hodzic2016}, that takes into account recent findings regarding
high formation rates for SOA, and an alternative ``1.5-dimensional''
volatility basis set (VBS) scheme, based on \citet{Koo2014}, which
handles both oxygenation and fragmentation of primary and secondary OA
by atmospheric oxidation (chemical aging), in a simplified way.

\item
The EMEP 3D model and chemical pre-processing systems (GenChem)
were re-harmonised, making use of a new python-based system. See
Sect.~\ref{sec:GChem}.

\item
We have added `EQSAM4clim' as an option for aerosol thermodynamics. See
Sect.~\ref{sec:EQSAM}.

\item
A bug-fix for radiation (PAR) was also needed, with significant impact
on POD calculations for forests. See Sect.~\ref{sec:PAR}.

\item
New methods to specify emissions inputs were added, which allow more
flexible input of data from external sources. See Sect.~\ref{sec:EmisInp}.

\item
The vertical profiles of emission releases were modified, reflecting
the transition to a finer resolution of the modelling grid and as a
part of the general model development towards a flexible code

\item
The use of configuration files was again expanded, replacing some
of the earlier hard-coded methods to specify model setups. See
Sect.~\ref{sec:Config}.


\end{itemize}



\section{EmChem19 chemical mechanism}
\label{sec:EmChem19}

The new gas-phase chemistry scheme -- EmChem19 -- is a substantial
revision of the EmChem16 scheme \citep{R2017:ModDev}.  The reaction rates
in EmChem19 are updated to be consistent with the latest recommendations
from the IUPAC \citep{IUPAC2019,Atkinson2004,Atkinson2006}, and evaluated against
the latest Master Chemical Mechanism (MCM v3.3.1, 
\citealt[][and refs therein]{Jenkin2015}). In addition to
these updates (and bug fixes) some new gas-phase reactions have been
added and a few new chemical species have been included in the chemical
mechanism.  Major changes compared to EmChem16 include:

\begin{itemize}
\item
The addition of benzene and toluene as emitted species. 
\item
A revised (very simplified) aromatic chemistry -- tuned to give
reasonably good agreement with the Master Chemical Mechanism 
\citep[MCM, see][]{Jenkin2015} in box model simulations.

\item
Addition of more than 20 new reactions -- including a number of reactions
between peroxy and nitrate radicals and a new treatment of peroxy -
peroxy radical reactions, using the total  \ce{RO2} pool as a reactant.

\item
Changes of the rates and/or products of ca 45 reactions.

\item
Revised SOA formation from sesquiterpenes (SQT) -- in EmChem19 SQT
emissions are assumed to be equal to 5\% (by mass) of the monoterpene
emissions; in the model the emitted SQT is assumed to immediately form
non-volatile BSOA, with 17\% yield (mass based).

\item
A few reactions were removed or simplified compared to EmChem16 -- this
includes a simplification of the minimal scheme for monoterpene chemistry
(based on \citealp{LamarqueCAMchem2012}), which was introduced in EmChem16;
in the standard EmChem19 scheme monoterpenes are modelled using a single
surrogate MT (APINENE) instead of the three used in EmChem16.

\end{itemize}
 
A detailed description of the EmChem19 scheme, including evaluation
against ambient measurements and comparison to the MCM and other
chemical mechanisms in box model simulations, will be given by
Bergstr{\"o}m et al., 2019.

\nocite{BergstromEmChem2019}


%For own info.
%Notes 20 Aug 2018
%    The SQT_SOA_NV yield in the “Nature-paper” runs seem very high? And the molecular formula for SQT_SOA_NV seems strange -- C15H22!? Gives a very low OM/OC ratio (and SQT actually is C15H24) -- using the direct emission of SQT_SOA_NV of 0.01534*MT means that the SQT-emissions (if we assume 17% mass yield from SQT) are assumed to be about 0.1338*MT emissions (and not 5%, which we have assumed earlier)
%    I will change to my old molecular formula for SQT_SOA_NV (SESQ_ng100) C14H22O7, with stated M of 302. And set the “direct” emission to 0.00383*MT to correspond to 5% SQT emissions (mass based) and 17% SOA-yield (mass based) 

%%%%%%%%%%%%%%%%%%%%%%%%%%%%%%%%%%%%%%%%%%%%%%%%%%%%%%%%%%%%%%%%%%%%%%%%%%%%%%
\section{Revised GenChem system}
\label{sec:GChem}.

The chemical mechanism files used in the EMEP CTM have for many years
been generated with a chemical pre-processor. Originally written in
perl, GenChem.pl, and now in python, GenChem.py, this pre-processor reads
files which describe the chemical mechanism using chemical notation (e.g.
krate   OH + NO2 = HNO3), and produces differential equations in fortran,
ready for import into the EMEP source code as the CM\_ files (e.g.
CM\_ChemSpecs\_mod.f90). GenChem.pl used
to be a part of the emepctm package as used at MSC-W, but now GenChem.py
is part of a separate system which allows testing of chemical mechanisms
with either a box-model (BoxChem) or the 1-D Ecosystem Surface
Exchange model (ESX, \citealt{R2014:ESX}). With this system the
EMEP model chemistry can be evaluated against more advanced and
well-evaluated schemes such as the Master Chemical Mechanism (MCM, e.g.
\citealt{Jenkin2015,Saunders2003a}, CRI \citep{JenkinCRI2019} % add more
 or carbon-bond \citep[]{Yarwood_et_al:2005,LueckenAE2019}. 


%%%%%%%%%%%%%%%%%%%%%%%%%%%%%%%%%%%%%%%%%%%%%%%%%%%%%%%%%%%%%%%%%%%%%%%%%%%%%%
\section{EQSAM4clim}
\label{sec:EQSAM}

Three thermodynamic equilibrium models are currently implemented in
the EMEP MSC-W model to calculate gas/aerosol partitioning of
inorganic semi-volatile species. The alternative models are:
Ammonium (EMEP simplified equilibrium scheme, \citealt{Hov:NOx}), MARS
\citep{BinkowskiShankar:1995}, and EQSAM \citep{Metzger2007}. The
MARS scheme, presently used in the standard model, has some
limitations, the major one being the number of chemical species
included in the thermodynamic equlibrium, but also the assumption of
metastable aqueous aerosols. This year, a new version of EQSAM -
EQSAM4clim \citep{Metzger:2016} - has been implemented as an
optional thermodynamic scheme. The advantage of the scheme is that it
considers a full gas-liquid-solid partitioning and can account
for more cations (e.g. Na$^{+}$, Ca$^{2+}$, Mg$^{2+}$) and anions
(Cl$^{-}$). Preliminary testing of EQSAM4clim has been carried out,
and the results are presented in Chapter~\ref{ch:eqsam4clim}. Further
evaluation of the scheme with more observations, as well as its
testing in source-receptor calculations are needed before the
EQSAM4clim can be considered for use in EMEP MSC-W reporting runs.

%%%%%%%%%%%%%%%%%%%%%%%%%%%%%%%%%%%%%%%%%%%%%%%%%%%%%%%%%%%%%%%%%%%%%%%%%%%%%%

\section{Radiation issues}
\label{sec:PAR}

% LandInputs_Apri2017 has light
% IAM_DF  0.006
% IAM_CR  0.0105
% and e.g. f ~  1.0 - exp(-0.006*300)  =  0.8347
% but           1.0 - exp(-0.0105*300) =  0.9571

As noted in \citet{R2018:ModDev}, the radiation scheme of
\citet{WeissNorman1985} was introduced in order to give better estimates
of diffuse versus direct photosynthetically active radiation (PAR), and to fix a
bug in the calculations of these variables which had been identified in
previous model versions.  PAR is used in modelling both stomatal uptake and
biogenic VOC emissions.  Unfortunately this bug-fix contained itself a
bug in the units used in the stomatal uptake (DO3SE) module, so that calculated PAR levels were too low. This
has now been corrected.  As found and explained in  \citet{R2018:ModDev},
and illustrated in  Fig.~\ref{fig:PODbug}, the impact
of the change is mainly apparent for forests.
The net result of fixing this bug is to restore POD levels to very
similar levels to those seen in the rv4.15 version discussed in \citet{R2018:ModDev}.
As before, the correlation coefficient between new and old versions is very high ($\ge$0.996).

\begin{figure}
%\includegraphics*[width=8cm]{FIGS_UPDATES/PlotScat_SURF_ppb_O3.png}
%\includegraphics*[width=8cm]{FIGS_UPDATES/PlotScat_POD1_IAM_DF.png}
%\includegraphics*[width=8cm]{FIGS_UPDATES/PlotScat_POD3_IAM_CR.png}
\caption{Comparison of model versions rv4.33 and rv4.32
 for mean ozone (top-left), POD1 for IAM deciduous forests (top-right) and
 POD$_3$IAM for crops (bottom). The dashed line represents the 1:1 line.
 Calculations are for the year 2012, using the 50km version of the model.
  \label{fig:PODbug}}
\end{figure}



%\QUERY{
%FROM R2018: 
% ..... Although this change has very limited
% impact on most results and pollutants, calculations of photo-toxic ozone
% dose (POD) were found to be rather large in some case, especially for
% forests. This is illustrated in  Fig.~\ref{fig:PODbug}, which compares
% results from two model runs using identical emissions and meteorology,
% but  versions rv4.15 and rv4.17. It can be seen that ozone itself is
% hardly affected by this change, but POD1 values for deciduous forests
% are about 30\% lower with
%rv4.17 than with rv4.15.  In both cases the two
%model versions correlate extremely well.
%WILL CONTINUE 19th August
%}

%At first sight, the lack of sensitivity of POD$_3$IAM to this problem seem surprising
%because higher thresholds ($Y$ in POD$Y$) tend to lead to greater sensitivity \citep{Tuovinen:EP2007}.
%However, the light response coefficients used in the calculation of stomatal conductance (\gsto)
%are quite different for crops and forests, such that \gsto for forests is  more likely to
%be  limited by low PAR values than crops. Further, the accumulation season for POD$_3$IAM in
%crops is shorter (90 days for IAM\_CR)
%and confined to the summer period when light levels are not limiting, whereas 
%the accumulation season for POD1 extends into the spring and autumn and thus
%includes more periods when light-levels act to limit \gsto.
%The impacts of this change will be investigated in more detail in the coming months.

%Fortunately, given that calculations of POD$_3$IAM have been more extensively
%used  than POD1 for  impact assessment \citep{MillsGCB2018a,MillsGCB2018b},
% as of these


%\begin{figure}
%\includegraphics*[width=8cm]{FIG_UPDATES/PlotScat_SURF_ppb_O3.png}
%\includegraphics*[width=8cm]{FIG_UPDATES/PlotScat_POD1_IAM_DF.png}
%\includegraphics*[width=8cm]{FIG_UPDATES/PlotScat_POD3_IAM_CR.png}
%\caption{Comparison of model versions rv4.15 and rv4.17
% for mean ozone (top-left), POD1 for IAM deciduous forests (top-right) and
% POD$_3$IAM for crops (bottom). The dashed line represents the 1:1 line.
% Calculations are for the year 2012, using the 50km version of the model.
%  \label{fig:PODbug}}
%\end{figure}

%%%%%%%%%%%%%%%%%%%%%%%%%%%%%%%%%%%%%%%%%%%%%%%%%%%%%%%%%%%%%%%%%%%%%%%%%%%%%%
\section{Emission inputs}
\label{sec:EmisInp}

A new system for reading emissions has been implemented. The new system allows
to read emissions in NetCDF, without the need of heavy preprocessing of these inputs.
Any 2-dimensional emission field can be directly used by the model, provided
the longitude and latitude data is provided. The metadata, such as species, units,
names, sectors, etc. can be provided separately in the configuration file,
if they are not found in the emission file.
The new system is at present implemented in addition to the previous one, and
emissions in both formats can be used at the same time.

An emission mask can be defined using any field in a NetCDF file. This mask can then
be used to reduce emission over the region covered by the mask.

%%%%%%%%%%%%%%%%%%%%%%%%%%%%%%%%%%%%%%%%%%%%%%%%%%%%%%%%%%%%%%%%%%%%%%%%%%%%%%
\section{Emission heights}
\label{sec:EmisHeight}

As documented in ~\citet{R2016}, the release heights of GNFR sectoral
emissions are assigned based on those previously developed for the SNAP
system. The heights of emission release are now expressed in
pressure levels, as documented in Table ~\ref{tab:emisH}. Here, the
P-levels indicate air pressure at the top of the layer, within which
the given fraction of emissions is released. The lowest layer (second
column) is confined between 101325.0 Pa (the surface) and 101084.9 Pa
(appr. 20m), etc. The vertical profiles of emission releases have been
slightly modified, using the experiences of EMEP
~\citep{Simpson_et_al:EMEP} and EuroDelta ~\citep{ED3:2014}.



\begin{table}[!h]
\begin{center}
\begin{footnotesize}
%\footnotesize    
%DS \begin{tabular}{|l|l|l|l|l|l|l|l|}
\caption{Definition used for the fractions of emissions, released within a given vertical
  layer, defined by pressure (P) levels. The P-levels are pressure (Pa) at
  the top of corresponding layer (P Surface = 101325.0 Pa). The
  approximate heights ($\Delta$Z, m) of the layer boundaries are also
  included.
\label{tab:emisH}
}
\begin{tabular}{llllllll}
\hline 
P-level &101084.9 &100229.1 &99133.2 &97489.35 &95206.225 &92283.825 &88722.15 \\
\hline
$\Delta$Z (m)& 0 - 20 & 20 - 92 & 92 - 184 & 184 - 324 & 324 - 522 & 522-781 & 781-1106\\
\hline
 GNFR 1  &0.0 &0.00  &0.0025 &0.1475 &0.40 &0.30 &0.15 \\
 GNFR 2  &0.06 &0.16 &0.75  &0.03  &0.00  &0.00  &0.0  \\
 GNFR 3  &1.0 &0.00  &0.00  &0.00  &0.00  &0.00  &0.0  \\
 GNFR 4  &0.05 &0.15 &0.70  &0.10  &0.00  &0.00  &0.0  \\
 GNFR 5  &1.0 &0.00  &0.00  &0.00  &0.00  &0.00  &0.0  \\
 GNFR 6  &1.0 &0.00  &0.00  &0.00  &0.00  &0.00  &0.0  \\
 GNFR 7  &1.0 &0.00  &0.00  &0.00  &0.00  &0.00  &0.0  \\
 GNFR 8  &1.0 &0.00  &0.00  &0.00  &0.00  &0.00  &0.0  \\
 GNFR 9  &1.0 &0.00  &0.00  &0.00  &0.00  &0.00  &0.0  \\
 GNFR 10 &0.0 &0.00  &0.41  &0.57  &0.02  &0.00  &0.0  \\
 GNFR 11 &1.0 &0.00  &0.00  &0.00  &0.00  &0.00  &0.0  \\
 GNFR 12 &1.0 &0.00  &0.00  &0.00  &0.00  &0.00  &0.0  \\
 GNFR 13 &0.02 &0.08 &0.60  &0.30  &0.00  &0.00  &0.0  \\
 \hline 
\end{tabular}
%\vspace{0.05in}
\end{footnotesize}
\end{center}
\end{table}



\begin{table}[!hb]
\begin{center}
\caption{The fractions of emissions, released within the actual model
  layers, for which the approximate heights ($\Delta$Z, m) of their
  boundaries are also shown.
\label{tab:ModelEmisH}}
\begin{footnotesize}
%\footnotesize    
%DS \begin{tabular}{|l|l|l|l|l|l|l|l|l|}
\begin{tabular}{lllllllll}
\hline
$\Delta$Z (m)& 0 - 50 & 50 - 94 & 94 - 155 & 155 - 237 & 237 - 341 & 341 - 623 & 623 - 1015& 1015 -1522\\
\hline
GNFR 1 & 0.000 &0.000 &0.002 &0.056 &0.125 &0.485 &0.290 &0.042\\
GNFR 2 &0.127  &0.111 &0.498 &0.245 &0.019 &0.000 &0.000 &0.000\\
GNFR 3& 1.000  &0.000 &0.000 &0.000 &0.000 &0.000 &0.000 &0.000\\
GNFR 4& 0.113  &0.104 &0.465 &0.256 &0.062 &0.000 &0.000 &0.000\\
GNFR 5& 1.000  &0.000 &0.000 &0.000 &0.000 &0.000 &0.000 &0.000\\
GNFR 6& 1.000  &0.000 &0.000 &0.000 &0.000 &0.000 &0.000 &0.000\\
GNFR 7& 1.000  &0.000 &0.000 &0.000 &0.000 &0.000 &0.000 &0.000\\
GNFR 8& 1.000  &0.000 &0.000 &0.000 &0.000 &0.000 &0.000 &0.000\\
GNFR 9& 1.000  &0.000 &0.000 &0.000 &0.000 &0.000 &0.000 &0.000\\
GNFR 10& 0.000 &0.010 &0.272 &0.342 &0.357 &0.018 &0.000 &0.000\\
GNFR 11&1.000  &0.000 &0.000 &0.000 &0.000 &0.000 &0.000 &0.000 \\
GNFR 12& 1.000 &0.000 &0.000 &0.000 &0.000 &0.000 &0.000 &0.000\\
GNFR 13& 0.054 &0.061 &0.398 &0.300 &0.187 &0.000 &0.000 &0.000\\
\hline 
\end{tabular}
%\vspace{0.05in}
\end{footnotesize}
\end{center}
\end{table}

These emission vertical profiles are defined independently of
the model's actual vertical layers. 
Thus, the emissions layers do
not necesserily match model layers, and the emissions are then
redistributed into actual model layers.  For the model levels used in
the runs for this report, the final emission release vertical profiles
are shown in Table ~\ref{tab:ModelEmisH}. Note that compared to
earlier runs, the thickness of the lowest layer is now decreased from
92~m to about 50~m.

%%%%%%%%%%%%%%%%%%%%%%%%%%%%%%%%%%%%%%%%%%%%%%%%%%%%%%%%%%%%%%%%%%%%%%%%%%%%%%
\section{Configuration}
\label{sec:Config}

The configuration parameters of the model run is entirely steered through a fortran namelist.
The file can easily be edited, to provide alternative parameters and paths to the input data.
Some key changes are:

\begin{itemize}

\item
  Many small changes to make model configuration easier and more flexible; see the
  User Guide for further explanation of some new methods and possibilities. 

\item
  The `femis' file used to control emission changes per country and  sector was
developed further, and can now use country  codes (e.g. `SE') rather than 
numbers (e.g. 18) to specify the source to be modified.

\item
  The configuration file is now organized in one large namelist instead of several ones.
  This will reduce the chances for syntactic errors.

\end{itemize}

%%%%%%%%%%%%%%%%%%%%%%%%%%%%%%%%%%%%%%%%%%%%%%%%%%%%%%%%%%%%%%%%%%%%%%%%%%%%%%
\section{Nesting}

A full snapshot of the model concentrations can be saved at specific given dates,
in addition to the standard modes for periodically saving boundary conditions.
Typically those can be used to be used as initial conditions for a new run in
a finer grid.

%%%%%%%%%%%%%%%%%%%%%%%%%%%%%%%%%%%%%%%%%%%%%%%%%%%%%%%%%%%%%%%%%%%%%%%%%%%%%%
\section{Biomass burning emissions}

It is easy now to switch between different schemes for forest fires emissions
by simply specifying the source in the configuration file (FINN, GFAS or GFED).
No need to recompile the code.  The vertical distribution of those emission takes into account the layer thickness.

%%%%%%%%%%%%%%%%%%%%%%%%%%%%%%%%%%%%%%%%%%%%%%%%%%%%%%%%%%%%%%%%%%%%%%%%%%%%%%
\section{WRF}

When WRF meteorological input is used, the model height of the surface is
read from the parameter `HGT', instead of an ad hoc separate file (topography.nc).

The model will now treat correctly the singularity at the Poles in the `lon lat' projection.

%%%%%%%%%%%%%%%%%%%%%%%%%%%%%%%%%%%%%%%%%%%%%%%%%%%%%%%%%%%%%%%%%%%%%%%%%%%%%%
\section{Outputs}

The time stamp of the NetCDF output is now defined in the middle of the period,
instead of the end. 

%\section{Acknowledgments}
%
%Any needed this year?
%
%Thanks are due to Ferd Sauter and Roy Wichink Kruit from RIVM for first pointing out
%problems with the radiation formulation, to John Johansson (Chalmers) for
%spotting problems with snow fields in WRF, and to Massimo Vieno (CEH, Edinburgh)
%for pointing out various landcover and other issues with the model.  % = lon/lat 23/11/2017)


%%%%%%%%%%%%%%%%%%%%%%%%%%%%%%%%%%%%%%%%%%%%%%%%%%%%%%%%%%%%%%%%%%%%%%%%%%%%%
\begin{table}
\begin{footnotesize}
\caption{Summary of major EMEP MSC-W model versions from 2012--2017.
Extends Table S1 of \citealt{Simpson_et_al:EMEP}}
\label{tab:Updates}
\centering
\begin{tabular}{lp{11cm}l}
\hline
Version & Update                                        & Ref$^{(a)}$   \\
\hline
rv4.33  & Public domain (June 2019)           & This report      \\
        & EmChem19, PAR bug-fix, EQSAM4clim   &                  \\
rv4.32  & Used for EMEP course, April 2019    &    \\
rv4.30  & Moved to new GenChem-based system  &   \\
        &                                    &        \\
rv4.17a & Used for R2018. Small updates         & R2018      \\
rv4.17  & Public domain (Feb. 2018)                   & R2018 \\
        & Corrections in global land-cover/deserts; added
          'LOTOS' option for European \ce{NH3} emissions; corrections
          to snow cover  & R2018 \\
        & \\
rv4.16  & New radiation scheme (Weiss\&Norman); Added dry and wet deposition for \ce{N2O5};
         (Used for  \citealt{Stadtler2018,MillsGCB2018b}) & R2018   \\
        & \\
rv4.15  & EmChem16 scheme & R2017 \\
%    Sect.\ref{sec:GNFR}--\ref{ss:Splits} & R2016  \\
%
        & \\
rv4.14  & Updated chemical scheme & R2017       \\
%% rv4.13 + CRI was used in McFiggans. Difficult to describe combo
        & \\
rv4.12  & New  global land-cover and BVOC & R2017       \\
        & \\
rv4.10  &  Public domain (Oct. 2016)                 
         (Used for  \citealt{MillsGCB2018a}) &  R2016 \\
        & \\
rv4.9   & Updates for GNFR sectors, DMS, sea-salt, dust, \ce{S_A} and  $\gamma$, \ce{N2O5} & \\ 
        &                                                &\\
rv4.8   &  Public domain (Oct. 2015)                   & R2015\\
        & ShipNOx introduced.                          
         Used for EMEP HTAP2 model calculations, see
see acp special issue: \url{https://www.atmos-chem-phys.net/special_issue390.html}). Also for \citet{Jonson_et_al:2017}. \\
        &                                                &\\
rv4.7   & Used for reporting, summer 2015                % & R2015  \\
        : New calculations of aerosol surface area; 
        ; New gas-aerosol uptake and \ce{N2O5} hydrolysis rates %CONT & \\
%         (Used  $\gamma=0.002$, \ce{N2O5} hydrolysis, for reporting.)   &\\
        ; Added 3-D calculations pf aerosol extinction and AODs; %& \\
        ; Emissions - new flexible mechanisms for interpolation and merging sources %& \\
        ; Global - monthly emissions from ECLIPSE project % & \\
        ; Global -  LAI changes from LPJ-GUESS model % & \\
        ; WRF meteorology \citep{SkamarockKlemp2008} can now
     be used directly in EMEP model. & R2015 \\
        &                                                &\\
rv4.6   & Used for Euro-Delta SOA runs                   & R2015  \\
%QUERY        & Bug-fix for ammonium deposition & \\
       & Revised boundary condition treatments % & \\  % Vertical profiles
       ; ISORROPIA capability added & \\
       &                                                &\\
rv4.5   & Sixth open-source (Sep 2014)                   & R2014\\
       & Improved dust, sea-salt, SOA modelling          % &      \\
       ; AOD and extinction coefficient calculations  updated %& \\
       ; Data assimilation system added % & \\
       ; Hybrid vertical coordinates replace earlier sigma % & \\
       ; Flexibility of grid projection increased. & \\
%SKIP        & ?? Point sources, plume rise, data-assimilation\\
       &                                                &\\
rv4.4   & Fifth open-source (Sep 2013) %
       ; Improved dust and sea-salt modelling   %          &      \\
       ; AOD and extinction coefficient calculations added %  &\\
       ; gfortran compatibility improved            %      &      \\
                  & R2014, R2013\\
       &                                                &\\
rv4.3   & Fourth public domain (Mar. 2013)  %
       ; Initial use of namelists           %            & \\
       ; Smoothing of MARS results         %            & \\
       ; Emergency module for volcanic ash and other events% & \\
       ; Dust and road-dust options added as defaults % & \\
       ; Advection algorithm changed  % & \\ % \citet{CLAPP98}    & \\
             & R2013\\ 
        &                                                &\\
rv4.0   & Third public domain (Sep. 2012)                & R2013\\ 
        & As documented in \citet{Simpson_et_al:EMEP}    & \\
%v2011-06& Second public domain (Aug. 2011)                &\\ 
%rv3     & First public domain (Sep. 2008)                &\\ 
        &                                                &\\
\hline
\end{tabular}
Notes: (a) R2018 refers to EMEP Status report 1/2018, etc.
\end{footnotesize}
\end{table}
%%%%%%%%%%%%%%%%%%%%%%%%%%%%%%%%%%%%%%%%%%%%%%%%%%%%%%%%%%%%%%%%%%%%%%%%%%%%%

\clearpage
\bibliographystyle{copernicus}         % change bibliography-name after each
\renewcommand\bibname{References}      % bibliographystyle command!
\addcontentsline{toc}{section}{References}
\bibliography{Refs,EMEP_Reports}
%\bibliography{myRefs,EMEP_Reports}
