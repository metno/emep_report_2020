\chapter*{OLD-OLD-OLD Acknowledgments}

\enlargethispage{3\baselineskip}

\vspace{-2cm}

This work has been funded by the EMEP Trust Fund.\\

The development of the EMEP MSC-W model has also been supported by Copernicus Atmosphere Modelling Service (CAMS) projects, the Nordic Council of Ministers, the Norwegian Space Centre and the Norwegian Ministry of Climate and Environment.
Development work has also been supported at Chalmers University of Technology in Sweden using funds from the Swedish Strategic Research project MERGE, the 
framework research program on `Photochemical smog in China' financed by
the Swedish Research Council (639-2013-6917), and FORMAS.

The work on condensable organics was partly funded by the Norwegian Ministry of Climate and Environment.
The work of TNO was funded
to a large extent by the Copernicus Atmosphere Monitoring Service (CAMS),
in particular the Contracts on emissions (CAMS\_81) and policy products
(CAMS\_71).\\


The work presented in this report has benefited largely from the work carried out under the four EMEP Task Forces and in particular under TFMM.\\

A large number of co-workers in participating countries have contributed in submitting quality assured data. The EMEP centers would like to express their gratitude for continued good co-operation and effort. The institutes and persons providing data are listed in the EMEP/CCC's data report and identified together with the data sets in the EBAS database. \\

For developing standardized methods, harmonization of measurements and improving the reporting guidelines and tools, the close co-operations with participants in the European Research Infrastructure for the observation of Aerosol, Clouds, and Trace gases (ACTRIS) as well as with the Scientific Advisory Groups (SAGs) in WMO/GAW are especially appreciated. \\


%Chris Heyes and Zig Klimont from EMEP CIAM/IIASA are acknowledged for provision of emission data on EC/OC and helpful discussions and advice.\\


The Working Group on Effects and its ICPs and Task Forces are
acknowledged for their assistance in determining the risk of damage
from air pollution.\\

The computations were partly performed on resources provided by 
UNINETT Sigma2 - the National Infrastructure for High Performance Computing and 
Data Storage in Norway (grant NN2890k and NS9005k). IT infrastructure in general was available through the Norwegian Meteorological Institute (MET Norway). Furthermore, the CPU time granted on the
\newpage
\noindent supercomputers owned by MET Norway has been of crucial importance for this year's source-receptor matrices and trend calculations. The CPU time made available by ECMWF to generate meteorology has been important for both the source-receptor and status calculations in this year's report.\\


%%% Local Variables: 
%%% mode: latex
%%% TeX-master: "report"
%%% End: 
